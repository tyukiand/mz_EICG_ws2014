\documentclass{scrartcl}

\usepackage{amsmath}	  % required for math in general
\usepackage{amsthm}     % environments for theorems, qed's etc
                        % (loaded after amsmath)
\usepackage{amssymb}	  % doublestroke symbols, other mathematical symbols
%\usepackage{dsfont}     % required for double-stroke 1 as characteristic function
\usepackage{array}	    % control of matrices and tables
\usepackage{graphicx}   % images

\usepackage{enumitem}   % more fine-grained control over enumerations
\setdescription{leftmargin=\parindent,labelindent=\parindent}

\usepackage{listings} % code listings
\lstset{basicstyle=\ttfamily\scriptsize}

% \input{diagrams.sty} (no category theory this time)

\usepackage{helvet}   % use (much fresher looking) helvetica for everything
\renewcommand{\familydefault}{\sfdefault}

\usepackage[weather]{ifsym}      % \Lightning symbol
% \usepackage{mathabx}             % \Asterisk causes some conflicts

% forcing the fucking floats to stop fucking floating like a fucking piece of
% shit in an ocean of fucking shit
\renewcommand{\topfraction}{.85}
\renewcommand{\bottomfraction}{.7}
\renewcommand{\textfraction}{.15}
\renewcommand{\floatpagefraction}{.66}
\renewcommand{\dbltopfraction}{.66}

% making all references into hyperlinks
\usepackage[dvipsnames]{xcolor}
\usepackage{hyperref}

\hypersetup{colorlinks=true,linkcolor=MidnightBlue,pdfborderstyle={/W 0}}

%\usepackage{anyfontsize}
\usepackage{datetime}

% forall
\let\oldforall\forall
\renewcommand{\forall}{\oldforall\,}

% parentheses
\newcommand{\rPar}[1]{\left(#1\right)} % round parens
\newcommand{\sPar}[1]{\left[#1\right]} % square parens
\newcommand{\cPar}[1]{\left\{#1\right\}} % curved parens 
\newcommand{\aPar}[1]{\left\langle #1 \right\rangle} % angle brackets

% floor and ceiling
\newcommand{\floor}[1]{{\left\lfloor#1\right\rfloor}} % curved parens 
\newcommand{\ceil}[1]{{\left\lceil#1\right\rceil}} % curved parens 

% norms
\newcommand{\abs}[1]{\left\lvert #1\right\rvert}
\newcommand{\norm}[1]{\left\lVert #1\right\rVert}
\newcommand{\scalar}[2]{\left\langle#1,#2\right\rangle}
\newcommand{\cross}{\times}
\DeclareMathOperator{\diam}{diam}
\DeclareMathOperator{\B}{B}

% intervals
\newcommand{\openOpenInterval}[2]{{\left(#1,#2\right)}}
\newcommand{\openClosedInterval}[2]{{\left(#1,#2\right]}}
\newcommand{\closedOpenInterval}[2]{{\left[#1,#2\right)}}
\newcommand{\closedClosedInterval}[2]{{\left[#1,#2\right]}}

% restriction of functions
\newcommand{\restrict}[2]{{\left.#1\right\vert_{#2}}}

% numbers
\newcommand{\Natural}{\mathbb{N}}
\newcommand{\Integer}{\mathbb{Z}}
\newcommand{\Real}{\mathbb{R}}
\newcommand{\Rational}{\mathbb{Q}}
\newcommand{\PositiveReal}{\Real_{>0}}
\newcommand{\NonnegativeReal}{\Real_{\geq0}}
\newcommand{\Complex}{\mathbb{C}}
\renewcommand{\i}{i}
\newcommand{\Quaternion}{\mathbb{H}}
\newcommand{\Boolean}{\mathbb{B}}

% function spaces
\newcommand{\SemiLebesgue}{\mathscr{L}}
\newcommand{\Continuous}{C}
\newcommand{\Lebesgue}{L}
\newcommand{\Sobolev}{H}
\newcommand{\Hilbert}{\mathscr{H}}
\newcommand{\Schwarz}{\mathscr{S}}

% set
\newcommand{\setPredicate}[2]{{\left\{#1\,\left\vert\, #2\right.\right\}}}
\newcommand{\set}[1]{{\left\{#1\right\}}}
\newcommand{\cardinality}[1]{\left\lvert #1 \right\rvert}
\newcommand{\powerset}[1]{\mathfrak{P}\left(#1\right)}
\DeclareMathOperator*{\intersection}{\bigcap}
\DeclareMathOperator*{\union}{\bigcup}
\newcommand{\disjointUnion}{\biguplus}
\renewcommand{\complement}[1]{#1^c}
% \newcommand{\setminus}{\backslash}
\newcommand{\injective}{\hookrightarrow}
\newcommand{\surjective}{\twoheadrightarrow}
%\DeclareMathOperator{\ker}{ker} % already exists... im does not?
\DeclareMathOperator{\im}{im}

% topological operators
\DeclareMathOperator{\Cl}{Cl}
\newcommand{\Closure}[2]{\Cl_{#1}\left(#2\right)}
\DeclareMathOperator{\const}{const}

% span and conv
\DeclareMathOperator*{\conv}{conv}
\DeclareMathOperator*{\linhull}{span}

% matrices
\newcommand{\mat}[2]{\left[\begin{array}{#1}#2\end{array}\right]}
\DeclareMathOperator*{\diag}{diag}

% landau symbols
\newcommand{\LandauO}[1]{\mathcal{O}\left(#1\right)}

% derivatives
\newcommand{\dd}[2]{\frac{\partial #1}{\partial #2}}
\newcommand{\differential}[1]{\boldsymbol{D}_{#1}}

% integrals
\renewcommand{\d}{\quad\mathrm{d}}

% characteristic functions, expected values, variances, covariances
% stochastic stuff
\newcommand{\one}[1]{\mathds{1}_{#1}}
\newcommand{\weakconv}[1]{\overset{#1}{\Longrightarrow}}
\newcommand{\wlim}{\mathop{\mathrm{wlim}}}
\newcommand{\vlim}{\mathop{\mathrm{vlim}}}

% lim inf lim sup
% \DeclareMathOperator{\liminf}{lim inf}
% \DeclareMathOperator{\limsup}{lim sup}

% qed etc.
\renewcommand{\qedsymbol}{$\blacksquare$}
\newcommand{\result}{\hfill $\Diamond$}

% lattices
\newcommand{\meet}{\wedge}
\newcommand{\join}{\vee}
\newcommand{\negate}{\neg}

% listings: Scala
\lstdefinelanguage{scala}{
  morekeywords={abstract,case,catch,class,def,%
    do,else,extends,false,final,finally,%
    for,if,implicit,import,match,mixin,%
    new,null,object,override,package,%
    private,protected,requires,return,sealed,%
    super,this,throw,trait,true,try,%
    type,val,var,while,with,yield},
  otherkeywords={=>,<-,<\%,<:,>:,\#,@},
  sensitive=true,
  morecomment=[l]{//},
  morecomment=[n]{/*}{*/},
  morestring=[b]",
  morestring=[b]',
  morestring=[b]"""
}
\lstset{showstringspaces=false}

% making references look a little nices
\let\oldRef\ref
\renewcommand{\ref}[1]{(\oldRef{#1})}

% weird stuff for computer science
\DeclareMathOperator{\arity}{ar}

% cat, category theory
% Bunch of categories
\DeclareMathOperator{\Id}{Id}
\DeclareMathOperator{\Top}{Top}
\DeclareMathOperator{\hTop}{h-Top}
\DeclareMathOperator{\Sets}{Sets}
\DeclareMathOperator{\Rel}{Rel}
\DeclareMathOperator{\FinSets}{FinSets}
\DeclareMathOperator{\Grp}{Grp}
\DeclareMathOperator{\Cat}{Cat}
\DeclareMathOperator{\Grpd}{Grpd}
\newcommand{\cat}[1]{\mathcal{#1}}
\newcommand{\Obj}{\mathrm{Obj}}
\newcommand{\Hom}{\mathrm{Hom}}
\newcommand{\op}{\mathrm{op}}
\newcommand{\nat}{\xrightarrow{\bullet}}
\newcommand{\iso}{\cong}
\newcommand{\dom}{\mathrm{dom}}
\newcommand{\cod}{\mathrm{cod}}
\DeclareMathOperator{\coeq}{Coeq}
\newcommand{\fst}{\mathrm{fst}}
\newcommand{\snd}{\mathrm{snd}}
\DeclareMathOperator{\Aut}{Aut}
\DeclareMathOperator{\End}{End}

% functors frequently used in various contexts
\DeclareMathOperator{\Free}{Free}
\DeclareMathOperator{\Forget}{Forget}

% empty set that is round
\let\emptyset\varnothing

% generated groups
\newcommand{\gen}[1]{\left\langle#1\right\rangle}
\newcommand{\normalSub}{\triangleleft}
\newcommand{\Asterisk}{\mathop{\scalebox{1.5}{\raisebox{-0.2ex}{$\ast$}}}}
\newcommand{\Sym}{\mathrm{Sym}}

% argmax argmin argsup etc.
\DeclareMathOperator{\argsup}{argsup}

% number theoretic operators
\DeclareMathOperator{\lcm}{lcm}

% get rid of the ugly-looking "epsilon"
\renewcommand{\epsilon}{\varepsilon}

% get rid of the empty-looking "angle"
\renewcommand{\angle}{\measuredangle}

\newcommand{\exercise}[2]{\vspace{1em}\noindent{\bf Exercise #1 (#2)}}
\renewcommand{\proof}{\vspace{0.8em}\noindent{\bf Proof: }}

\begin{document}
\noindent{\footnotesize Computer Graphics 2014/15, Exercise 6} 
\hfill 
{\footnotesize 14.01.2015}
\newline
{\footnotesize \input{../../NAMES.txt}}

\noindent\hrulefill

\exercise{6.1}{Trackball}

\noindent {\bf a)} 
We want to derive an expression for a quaternion that
corresponds to the rotation of a unit vector $u$
to some other unit vector $v$ (assuming that $u\neq \pm v$).

We can simply choose 
\[
  z := \frac{u^T v}{\norm{u \cross v}}
\] 
as rotation axis and compute the angle as follows:
\[
  \phi := \arccos(u^T v).
\]
Then we can simply plug the both numbers into the formula for the rotation 
quaternion:
\[
  \rPar{\cos\rPar{\frac{\phi}{2}}, \sin\rPar{\frac{\phi}{2}} z}.
\]

\noindent {\bf b)} The above formula would probably do the job, however, we
might want to eliminate unnecessary normalizations and trigonometric functions.
For this, remember that it holds (for angles from $[0, \pi]$):
\[
  \cos\rPar{\frac{\phi}{2}} = \sqrt{\frac{1 + \cos \phi}{2}}
  \qquad
  \sin\rPar{\frac{\phi}{2}} = \sqrt{\frac{1 - \cos \phi}{2}}.
\]
Furthermore, notice that $\cos(\phi) = u^T v$ and 
\[
  \norm{u \cross v} = \sin(\phi) = \sqrt{1 - (u^T v)^2}.
\]
Therefore it holds (for $q$ from the first part):
\begin{align*}
  \sqrt{2}\sqrt{1 + u^T v} q 
  &= 
  \sqrt{2}\sqrt{1 + u^T v}\rPar{
    \sqrt{\frac{1 + u^T v}{2}},
    \frac{u\cross v}{\sqrt{1 - (u^T v)^2}}
    \sqrt{\frac{1 - u^T v}{2}}
  } \\
  &= (1 + u^T v, u\cross v),
\end{align*}
so we see that we can compute $(1 + u^T v, u\cross v)$ and normalize it, 
instead of messing with trigonometric functions.

\exercise{6.2}{Converting into 	Euler's angles}
\begin{enumerate}[label={a)}]
	\item	
\end{enumerate}	
	Angles have negative sense of rotation.\\
	$R_x(\phi)=\begin{pmatrix}
			1& 0& 0\\
			0& \cos(\phi)& \sin(\phi)\\
			0& -\sin(\phi)& \cos(\phi)
		\end{pmatrix}$
	$R_y(\theta)=\begin{pmatrix}
		\cos(\theta)& 0& -\sin(\theta)\\
		0& 1& 0\\
		\sin(\theta)& 0 & \cos(\theta)
	\end{pmatrix}$
	\\
	$R_z(\psi)=\begin{pmatrix}
		\cos(\psi)& \sin(\psi)& 0\\
		-\sin(\psi)& \cos(\psi)& 0\\
		0& 0& 1
	\end{pmatrix}$
	
	\begin{flalign*}
	 & R_x(\phi)R_y(\theta)R_z(\psi)
	\\&=\begin{pmatrix}
		1& 0& 0\\
		0& \cos(\phi)& \sin(\phi)\\
		0& -\sin(\phi)& \cos(\phi)
	\end{pmatrix}
	\begin{pmatrix}
		\cos(\theta)& 0& -\sin(\theta)\\
		0& 1& 0\\
		\sin(\theta)& 0 & \cos(\theta)
	\end{pmatrix}
	\begin{pmatrix}
		\cos(\psi)& \sin(\psi)& 0\\
		-\sin(\psi)& \cos(\psi)& 0\\
		0& 0& 1
	\end{pmatrix}\\&=
	\begin{pmatrix}
	1& 0& 0\\
	0& \cos(\phi)& \sin(\phi)\\
	0& -\sin(\phi)& \cos(\phi)
	\end{pmatrix}
	\begin{pmatrix}
		\cos(\theta)\cos(\psi)& \cos(\theta)\sin(\psi)& -\sin(\theta)\\
		-\sin(\psi)& \cos(\psi)& 0\\
		\sin(\theta)\cos(\psi)& \sin(\theta)\sin(\psi) & \cos(\theta)
	\end{pmatrix}
	\\&=	\begin{pmatrix}
	\cos(\theta)\cos(\psi)& \cos(\theta)\sin(\psi)& -\sin(\theta)\\
	\sin(\phi)\sin(\theta)\cos(\psi)-\cos(\phi)\sin(\psi)& \sin(\phi)\sin(\theta)\sin(\psi)+\cos(\phi)\cos(\psi)& \sin(\phi)\cos(\theta)\\
	\cos(\phi)\sin(\theta)\cos(\psi)+\sin(\phi)\sin(\psi)& \cos(\phi)\sin(\theta)\sin(\psi)-\sin(\phi)\cos(\psi) & \cos(\phi)\cos(\theta)
	\end{pmatrix}
	\end{flalign*}

\begin{enumerate}[label={b)}]
	\item
	$R_x(\phi)R_y(\theta)R_z(\psi)=\begin{pmatrix}
		x_1 & y_1 & z_1\\
		x_2 & y_2 & z_2\\
		x_3 & y_3 & z_3
	\end{pmatrix}$
	\\$\Rightarrow$ 
	\begin{equation}
		\theta=-\arcsin(z_1)
		\label{theta}
	\end{equation}
	\begin{itemize}
	\item  $\theta\neq\pm \frac{\pi}{2}$:
	\begin{equation}
		y_1=\sin(\psi)\cos(\theta)
		\label{psi}
	\end{equation}
	\begin{equation}
		z_2=\sin(\phi)\cos(\theta)
		\label{phi}
	\end{equation} 
	\begin{equation}
	\cos(\arcsin(x))=\sqrt{1-\sin^2(\arcsin(x))}=\sqrt{1-x^2}
	\label{cos}
	\end{equation}
	\\$\Rightarrow$ The solutions for $\psi$ and $\phi$ are the same except that where $y_1$ is in the solution for $\psi$ there is $z_2$ in the solution for $\phi$.
	\\Equation \ref{psi},\ref{cos}
	$\Rightarrow$ 
	\begin{align}
		\psi &=\arcsin\left( \frac{y_1}{\cos(\theta)}\right) \nonumber\\
		\psi &=\arcsin\left( \frac{y_1}{\cos(\arcsin(z_1))}\right) \nonumber\\
		\psi &=\arcsin\left( \frac{y_1}{\sqrt{1-z_1^2}}\right) 
	\end{align}

	$\Rightarrow$
	 \begin{equation}
		\phi=\arcsin\left( \frac{z_2}{\sqrt{1-z_1^2}}\right)
	\end{equation}
	
	
	\item $\theta=\pm \frac{\pi}{2}$
	\\Addition theorems:
	\begin{equation}
	\sin(x\pm y)=\sin(x)\cos(y)\pm \cos(x)\sin(y)
	\label{addSin}
	\end{equation}
	
	\begin{equation}
	\cos(x\pm y)=\cos(x)\cos(y)\mp \sin(x)\sin(y)
	\label{addCos}
	\end{equation}
	 $\theta= \frac{\pi}{2}\Leftrightarrow\sin(\theta)=1$
	\\ With equation \ref{addCos},\ref{addSin} and $\cos(\theta)=0$	 you get the following matrix
	\begin{equation*}
		R=\begin{pmatrix}
			0& 0&-1\\
			\sin(\phi-\psi)& \cos(\phi-\psi)& 0\\
			\cos(\phi-\psi))&-\sin(\phi-\psi) & 0)
		\end{pmatrix}
	\end{equation*}
	$\Rightarrow$
	\begin{equation*}
		\phi-\psi=\arcsin(x_2)\\
	\end{equation*}
	\begin{equation}
		\phi=\arcsin(x_2)+\psi
	\end{equation}
		 $\theta= -\frac{\pi}{2}\Leftrightarrow\sin(\theta)=-1$
		 \\ With equation \ref{addCos},\ref{addSin} and $\cos(\theta)=0$ you get the following matrix
		 \begin{equation*}
		 	R=\begin{pmatrix}
		 		0& 0& 1\\
		 		-\sin(\phi+\psi)& \cos(\phi+\psi)& 0\\
		 		-\cos(\phi+\psi))&-\sin(\phi+\psi) & 0)
		 	\end{pmatrix}
		 \end{equation*}
		 $\Rightarrow$
		 \begin{equation*}
		 	\phi+\psi=\arcsin(-x_2)\\
		 \end{equation*}
		 \begin{equation}
		 \phi=-(\arcsin(x_2)+\psi)
		 \end{equation}
\end{itemize}

\end{enumerate}


\exercise{6.3}{Some basic properties of quaternions}

We denote 
\footnote{What's that ``underscore''-thingie, how does one TeX it?} quaternions as $\bar a = (a_0, a)$ and assume that the
vector-part $a$ is a column vector. We sometimes write the whole
quaternion vertically. Remember that the multiplication of quaternions
is defined as follows:
\[
  \bar a \cdot \bar b = (a_0 b_0 - a^T b, a_0 b + b_0 a + a \cross b).
\]

\noindent {\bf a)} We want to show that the multiplication of 
quaternions is associative (which is not obvious, because the quaternion product contains the cross product).
The brute-force proof with all $3$ quaternions simultaneously 
seems to be surprisingly nasty, so instead we embed the 
quaternions into a matrix algebra about which we know that it is 
associative (this allows us to deal with just two quaternions at a time).

First, we express quaternion multiplication as a matrix product:
\[
  \bar a \cdot \bar b = \mat{cc}{
    a_0 & -a^T \\
    a   & (a_0 I + [a]_\cross)
  }
  \mat{c}{b_0 \\ b},
\]
and write $M_{\bar a}$ for the 4x4-matrix.
It is not immediately obvious how the product of two such matrices relates to the product of quaternions. For the following 
computation we have to remember:
\begin{itemize}
  \item $[a \cross b]_\cross = b a^T - a b^T$ (This is the BAC-CAB rule in matrix form)
  \item $[a]_\cross [b]_\cross = b a^T - a^T b I$ (This is again the
  BAC-CAB rule, up to some sign changes and permutations)
\end{itemize} 
Using these BAC-CAB-rules in the lower right corner of the matrix, we compute:
\begin{align*}
  M_{\bar a} M_{\bar b}
    &= \mat{cc}{
      a_0 & -a^T \\
      a   & a_0 I + [a]_\cross
    }
    \mat{cc}{
      b_0 & -b^T \\
      b   & b_0 I + [b]_\cross
    } \\
    &= \mat{cc}{
      (a_0 b_0 - a^T b) & (-a_0 p^T - b_0 a^T - a^T [b]_\cross) \\
      (a_0 b + b_0 a + [a]_\cross b) &
        -a b^T + a_0 b_0 I + a_0 [b]_\cross + b_0 [a]_\cross + 
        [a]_\cross [b]_\cross
    } \\
    &= \mat{cc} {
      (a_0 b_0 - a^T b) & -(a_0 p^T + b_0 a^T + a \cross b) \\
      (a_0 b + b_0 a + [a]_\cross b) &
        (a_0 p_0 - a^T b)I + [a_0 b + b_0 a + a\cross b]_\cross
    } \\
    &= M_{\bar a \bar b}.
\end{align*}
Now using this, 
for arbitrary quaternions $\bar a, \bar b, \bar c$ we can 
easily verify:
\[
  (\bar a \bar b) \bar c = 
  \rPar{M_{\bar a \bar b}}\bar c =
  \rPar{M_{\bar a} M_{\bar b}} \bar c = 
  M_{\bar a} \rPar{M_{\bar b} \bar c} =
  \bar a (\bar b \bar c),
\]
because we already know that matrix product is associative.
\hfill \qed

\noindent {\bf b)} Despite being completely trivial, the
following computation is actually kind of important, 
because it shows us that $\bar a \bar a^\ast$ is
actually a real number:
\[
  \bar a \bar a^\ast = 
    (a_0^2 - \norm{a}^2, 0) 
    = \bar a^\ast \bar a
\]
This in turn allows us to define inverses of quaternions
analogously to the inverses of complex numbers.
\hfill \qed

\noindent {\bf c)} Notice: $-p\cross q = (-q) \cross (-p)$.
Therefore:
\begin{align*}
  (\bar p \bar q)^\ast 
    &= (p_0 q_0 - p^T q, -q_0 p - p_0 q - p\cross q) \\
    &= (q_0, -q)(p_0, -p) \\
    &= \bar q^\ast \bar p^\ast
\end{align*}
The same in matrix-notation introduced above (bars omitted):
\[
  M_{(a b)^\ast} = M^T_{ab} = \rPar{M_a M_b}^T = M_b^T M_a^T = 
  M_{b^\ast} M_{a^\ast}.
\]
\hfill \qed

\exercise{6.4}{Two-sheeted covering of $SO(3)$}
We want to show that for every unit quaternion $\bar q$
the rotation induced by $\bar q$ is the same as the 
rotation induced by $-\bar a$.

There are many possible ways.
The simplest one is as follows: quaternions commute
with all real numbers, in particular with $\pm 1$, 
therefore for each $\bar p = (0, v)$ it holds:
\[
  (-\bar q)\bar p (- \bar q)^\ast = 
  (-1)^2 \bar q \bar p \bar q^\ast.
  = \bar q \bar p \bar q^\ast.
\]

A somewhat less simple possibility is to take a look
at the resulting rotation matrix (for components
$\bar q = (a, b, c, d)$):
\[
  \mat{ccc}{
        a^{2} + b^{2} -c^{2} -d^{2} & -2 a d +2 b c & 2 a c +2 b d \\
        2 a d +2 b c &  a^{2} + c^{2} -b^{2} -d^{2} & -2 a b +2 c d \\
        -2 a c +2 b d & 2 a b +2 c d &  a^{2} + d^{2} -b^{2} -c^{2}
      } 
\]
All entries of this matrix are actually polynomials in $a, b, c, d$ consisting only of terms with degree exactly $2$, therefore the factor $(-1)^2$ just cancels out everywhere.

Another possibility is to consider the Rodrigues-rotation 
matrix for the angle/axis $(\phi, z)$ associated with the 
quaternion $\bar q$ and to remember that $\cos$ is symmetric
and $\sin$ antisymmetric:
\begin{align*}
  R_{(-\phi,-z)} 
    &= \cos(-\phi)I + \sin(-\phi)[-z]_\cross + 
       (1 - \cos(-\phi))(-z)(-z)^T \\
    &= \cos(\phi)I + (-1)^2\sin(\phi)[z]_\cross +
       (1 - \cos(\phi))(-1)^2 zz^T \\
    &= R_{(\phi, z)}
\end{align*}
\hfill \qed

\end{document}
