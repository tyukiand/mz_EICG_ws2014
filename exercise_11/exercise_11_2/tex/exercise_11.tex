\documentclass{scrartcl}

\input{./packages}
\input{./definitions}
\newcommand{\exercise}[2]{\vspace{1em}\noindent{\bf Exercise #1 (#2)}}
\renewcommand{\proof}{\vspace{0.8em}\noindent{\bf Proof: }}

\begin{document}
\noindent{\footnotesize Computer Graphics 2014/15, Exercise 11} 
\hfill 
%{\footnotesize \input{./currentDate.txt}}
\newline
{\footnotesize \input{../../NAMES.txt}}

\noindent\hrulefill

\exercise{11.1}{Normal transformation}
\begin{itemize}
\item[a)]
\begin{align*}
	n'\cdot t'=(n')^T t&\stackrel{!}{=}n^Tt=n\cdot t\\
	n^TG^TMt&\stackrel{!}{=}n^Tt\\
	\Rightarrow G^TM&=\mathbbm{1}\\
	G^T&=M^{-1}\\
	\Rightarrow \hspace*{6mm} G\hspace*{2.5mm}&=\left(M^{-1} \right)^T 
\end{align*}
\item[b)]
	The normal has to point outside, but if some components are inverted we could not assume that this is conserved.
\item[c)]
	View the ellipsoid as a dragged unit ball:\\
	\begin{itemize}
		\item[$\bullet$] ball\\
		$\begin{pmatrix}
			x\\y\\z
		\end{pmatrix}$ with $x^2+y^2+z^2 =1$\\
		\item[$\bullet$] ellipsoid\\
				$\begin{pmatrix}
				a x\\b y \\ c z
				\end{pmatrix}$ with $\left( \frac{x}{a}\right) ^2+\left( \frac{y}{b}\right) ^2+\left( \frac{z}{c}\right) ^2 =1$
	\end{itemize}
	In our case are $a=4$, $b=2$, $c=1$\\
	$\Rightarrow$ $M=\begin{pmatrix}
										4 & 0 & 0\\
										0 & 2 & 0\\
										0 & 0 & 1\\
									\end{pmatrix}$\\
	$\Rightarrow$ $G=\left( M^{-1}\right)^T =\begin{pmatrix}
																				\frac{1}{4}& 0 & 0\\
																				0 & \frac{1}{2} & 0\\
																				0 & 0 & 1\\
																				\end{pmatrix}$\\
	normals of ball: $n_B=\begin{pmatrix}
									x\\y\\z
								\end{pmatrix}$ \\
	normal od ellipsoid: $n_e=G n_B=	\begin{pmatrix}
															\frac{x}{4}\\
															 \frac{y}{2}\\
															z\\
														\end{pmatrix}$\\						
\end{itemize}
\exercise{11.2}{Cox-De Boor algorithm}
\\Out of the Lecture, the definition of $B_{i,j}(t)$
\begin{equation}
\label{db}
B_{i,j}(t)=\tfrac{t-t_i}{t_{i+j}-t_i}B_{i,j-1}(t)+\tfrac{t_{i+j}-t}{t_{i+j}-t_i}B_{i,j-1}(t)
\end{equation}
and $B_{i,0}(t)=\delta_{i,l}$ we could follow that $B_{i,j}(t)=0$ if $i>l$ or $i+j<l$. 
Now lets show that 
\begin{equation}
\sum_{i=l-d+k}^{l}B_{i,d-k}(t)p_i^k\stackrel{!}{=}\sum_{i=l-d+k+1}^{l}B_{i,d-k-1}(t)p_i^{k+1}
\end{equation}
We start using \ref{db}
\begin{align}
	\sum_{i=l-d+k}^{l}B_{i,d-k}(t)p_i^k
	&=\sum_{i=l-d+k}^{l}\left( \frac{t-t_i}{t_{i+d-k}-t_i}B_{i,d-k-1}(t)+\frac{t_{i+d-k+1}-t}{t_{i+d-k+1}-t_{i+1}}B_{i+1,d-k-1}(t)\right)p_i^k\\
	&=\sum_{i=l-d+k+1}^{l}B_{i,d-k-1}(t)\underbrace{\left( \frac{t-t_i}{t_{i+d-k}-t_i}p_i^k+\frac{t_{i+d-k}-t}{t_{i+d-k}-t_i}p_{i-1}^k\right)  }_{\substack{p_i^{k+1}}}\\
	&+\frac{t-t_{l-d+k}}{t_{l}-t_{l-d+k}}
	\underbrace{B_{l-d+k,d-k-1}}_{=0\ \forall d,k}(t)p_{l-d+k}^k+
	\frac{t_{l+d-k+1}-t}{t_{l+d-k+1}-t_{l+1}}
	\underbrace{B_{l+1,d-k-1}(t)}_{=0\ \forall d,k}p_l^k\\
	&=\sum_{i=l-d+k+1}^{l}B_{i,d-k-1}(t)p_i^{k+1}
\end{align}
We now insert $k=0$ and $k=d$ and get the expected formular by iteration over all $0<k<d$ as
\begin{equation}
x(t)=\sum_{i=l-d}^{l}B_{i,d}(t)p_i^0=...=\sum_{i=l}^{l}B_{i,0}(t)p_i^d=B_{l,0}(t)p_l^d=p_l^d
\end{equation}





\end{document}
