\documentclass{scrartcl}


\usepackage{amsmath}	  % required for math in general
\usepackage{amsthm}     % environments for theorems, qed's etc
                        % (loaded after amsmath)
\usepackage{amssymb}	  % doublestroke symbols, other mathematical symbols
%\usepackage{dsfont}     % required for double-stroke 1 as characteristic function
\usepackage{array}	    % control of matrices and tables
\usepackage{graphicx}   % images

\usepackage{enumitem}   % more fine-grained control over enumerations
\setdescription{leftmargin=\parindent,labelindent=\parindent}

\usepackage{listings} % code listings
\lstset{basicstyle=\ttfamily\scriptsize}

% \input{diagrams.sty} (no category theory this time)

\usepackage{helvet}   % use (much fresher looking) helvetica for everything
\renewcommand{\familydefault}{\sfdefault}

\usepackage[weather]{ifsym}      % \Lightning symbol
% \usepackage{mathabx}             % \Asterisk causes some conflicts

% forcing the fucking floats to stop fucking floating like a fucking piece of
% shit in an ocean of fucking shit
\renewcommand{\topfraction}{.85}
\renewcommand{\bottomfraction}{.7}
\renewcommand{\textfraction}{.15}
\renewcommand{\floatpagefraction}{.66}
\renewcommand{\dbltopfraction}{.66}

% making all references into hyperlinks
\usepackage[dvipsnames]{xcolor}
\usepackage{hyperref}

\hypersetup{colorlinks=true,linkcolor=MidnightBlue,pdfborderstyle={/W 0}}

%\usepackage{anyfontsize}
\usepackage{datetime}

% forall
\let\oldforall\forall
\renewcommand{\forall}{\oldforall\,}

% parentheses
\newcommand{\rPar}[1]{\left(#1\right)} % round parens
\newcommand{\sPar}[1]{\left[#1\right]} % square parens
\newcommand{\cPar}[1]{\left\{#1\right\}} % curved parens 
\newcommand{\aPar}[1]{\left\langle #1 \right\rangle} % angle brackets

% floor and ceiling
\newcommand{\floor}[1]{{\left\lfloor#1\right\rfloor}} % curved parens 
\newcommand{\ceil}[1]{{\left\lceil#1\right\rceil}} % curved parens 

% norms
\newcommand{\abs}[1]{\left\lvert #1\right\rvert}
\newcommand{\norm}[1]{\left\lVert #1\right\rVert}
\newcommand{\scalar}[2]{\left\langle#1,#2\right\rangle}
\newcommand{\cross}{\times}
\DeclareMathOperator{\diam}{diam}
\DeclareMathOperator{\B}{B}

% intervals
\newcommand{\openOpenInterval}[2]{{\left(#1,#2\right)}}
\newcommand{\openClosedInterval}[2]{{\left(#1,#2\right]}}
\newcommand{\closedOpenInterval}[2]{{\left[#1,#2\right)}}
\newcommand{\closedClosedInterval}[2]{{\left[#1,#2\right]}}

% restriction of functions
\newcommand{\restrict}[2]{{\left.#1\right\vert_{#2}}}

% numbers
\newcommand{\Natural}{\mathbb{N}}
\newcommand{\Integer}{\mathbb{Z}}
\newcommand{\Real}{\mathbb{R}}
\newcommand{\Rational}{\mathbb{Q}}
\newcommand{\PositiveReal}{\Real_{>0}}
\newcommand{\NonnegativeReal}{\Real_{\geq0}}
\newcommand{\Complex}{\mathbb{C}}
\renewcommand{\i}{i}
\newcommand{\Quaternion}{\mathbb{H}}
\newcommand{\Boolean}{\mathbb{B}}

% function spaces
\newcommand{\SemiLebesgue}{\mathscr{L}}
\newcommand{\Continuous}{C}
\newcommand{\Lebesgue}{L}
\newcommand{\Sobolev}{H}
\newcommand{\Hilbert}{\mathscr{H}}
\newcommand{\Schwarz}{\mathscr{S}}

% set
\newcommand{\setPredicate}[2]{{\left\{#1\,\left\vert\, #2\right.\right\}}}
\newcommand{\set}[1]{{\left\{#1\right\}}}
\newcommand{\cardinality}[1]{\left\lvert #1 \right\rvert}
\newcommand{\powerset}[1]{\mathfrak{P}\left(#1\right)}
\DeclareMathOperator*{\intersection}{\bigcap}
\DeclareMathOperator*{\union}{\bigcup}
\newcommand{\disjointUnion}{\biguplus}
\renewcommand{\complement}[1]{#1^c}
% \newcommand{\setminus}{\backslash}
\newcommand{\injective}{\hookrightarrow}
\newcommand{\surjective}{\twoheadrightarrow}
%\DeclareMathOperator{\ker}{ker} % already exists... im does not?
\DeclareMathOperator{\im}{im}

% topological operators
\DeclareMathOperator{\Cl}{Cl}
\newcommand{\Closure}[2]{\Cl_{#1}\left(#2\right)}
\DeclareMathOperator{\const}{const}

% span and conv
\DeclareMathOperator*{\conv}{conv}
\DeclareMathOperator*{\linhull}{span}

% matrices
\newcommand{\mat}[2]{\left[\begin{array}{#1}#2\end{array}\right]}
\DeclareMathOperator*{\diag}{diag}

% landau symbols
\newcommand{\LandauO}[1]{\mathcal{O}\left(#1\right)}

% derivatives
\newcommand{\dd}[2]{\frac{\partial #1}{\partial #2}}
\newcommand{\differential}[1]{\boldsymbol{D}_{#1}}

% integrals
\renewcommand{\d}{\quad\mathrm{d}}

% characteristic functions, expected values, variances, covariances
% stochastic stuff
\newcommand{\one}[1]{\mathds{1}_{#1}}
\newcommand{\weakconv}[1]{\overset{#1}{\Longrightarrow}}
\newcommand{\wlim}{\mathop{\mathrm{wlim}}}
\newcommand{\vlim}{\mathop{\mathrm{vlim}}}

% lim inf lim sup
% \DeclareMathOperator{\liminf}{lim inf}
% \DeclareMathOperator{\limsup}{lim sup}

% qed etc.
\renewcommand{\qedsymbol}{$\blacksquare$}
\newcommand{\result}{\hfill $\Diamond$}

% lattices
\newcommand{\meet}{\wedge}
\newcommand{\join}{\vee}
\newcommand{\negate}{\neg}

% listings: Scala
\lstdefinelanguage{scala}{
  morekeywords={abstract,case,catch,class,def,%
    do,else,extends,false,final,finally,%
    for,if,implicit,import,match,mixin,%
    new,null,object,override,package,%
    private,protected,requires,return,sealed,%
    super,this,throw,trait,true,try,%
    type,val,var,while,with,yield},
  otherkeywords={=>,<-,<\%,<:,>:,\#,@},
  sensitive=true,
  morecomment=[l]{//},
  morecomment=[n]{/*}{*/},
  morestring=[b]",
  morestring=[b]',
  morestring=[b]"""
}
\lstset{showstringspaces=false}

% making references look a little nices
\let\oldRef\ref
\renewcommand{\ref}[1]{(\oldRef{#1})}

% weird stuff for computer science
\DeclareMathOperator{\arity}{ar}

% cat, category theory
% Bunch of categories
\DeclareMathOperator{\Id}{Id}
\DeclareMathOperator{\Top}{Top}
\DeclareMathOperator{\hTop}{h-Top}
\DeclareMathOperator{\Sets}{Sets}
\DeclareMathOperator{\Rel}{Rel}
\DeclareMathOperator{\FinSets}{FinSets}
\DeclareMathOperator{\Grp}{Grp}
\DeclareMathOperator{\Cat}{Cat}
\DeclareMathOperator{\Grpd}{Grpd}
\newcommand{\cat}[1]{\mathcal{#1}}
\newcommand{\Obj}{\mathrm{Obj}}
\newcommand{\Hom}{\mathrm{Hom}}
\newcommand{\op}{\mathrm{op}}
\newcommand{\nat}{\xrightarrow{\bullet}}
\newcommand{\iso}{\cong}
\newcommand{\dom}{\mathrm{dom}}
\newcommand{\cod}{\mathrm{cod}}
\DeclareMathOperator{\coeq}{Coeq}
\newcommand{\fst}{\mathrm{fst}}
\newcommand{\snd}{\mathrm{snd}}
\DeclareMathOperator{\Aut}{Aut}
\DeclareMathOperator{\End}{End}

% functors frequently used in various contexts
\DeclareMathOperator{\Free}{Free}
\DeclareMathOperator{\Forget}{Forget}

% empty set that is round
\let\emptyset\varnothing

% generated groups
\newcommand{\gen}[1]{\left\langle#1\right\rangle}
\newcommand{\normalSub}{\triangleleft}
\newcommand{\Asterisk}{\mathop{\scalebox{1.5}{\raisebox{-0.2ex}{$\ast$}}}}
\newcommand{\Sym}{\mathrm{Sym}}

% argmax argmin argsup etc.
\DeclareMathOperator{\argsup}{argsup}

% number theoretic operators
\DeclareMathOperator{\lcm}{lcm}

% get rid of the ugly-looking "epsilon"
\renewcommand{\epsilon}{\varepsilon}

% get rid of the empty-looking "angle"
\renewcommand{\angle}{\measuredangle}

\DeclareMathOperator{\argmin}{argmin}
\DeclareMathOperator{\dist}{dist}
\newcommand{\exercise}[2]{\vspace{1em}\noindent{\bf Exercise #1 (#2)}}
\renewcommand{\proof}{\vspace{0.8em}\noindent{\bf Proof: }}

\begin{document}
\noindent{\footnotesize Computer Graphics 2014/15, Exercise 3} 
\hfill 
{\footnotesize 14.01.2015}
\newline
{\footnotesize \input{../../NAMES.txt}}

\noindent\hrulefill

\exercise{3.1}{Polygon picking and snapping}
For the vertex-to-line snapping behavior we have to be able to project points 
to line segments.

Let $a, b, x\in \Real^2$ be three points, let $[a, b]$ denote the line segment
from $a$ to $b$. We want to find a point $p\in[a,b]$ such that
\[
  p = \argmin_{p^\prime \in [a, b]}\dist\rPar{p^\prime, x}.
\]
For this, we first move our coordinate system to $a$:
\begin{align*}
  d := b - a \\
  r := x - a
\end{align*}
Now the the value $\xi := \scalar{\frac{d}{\norm{d}}}{r}$ is the signed 
distance along $d$ from $a$ to the projection of $x$ to the infinite line
through $a$ and $b$. 
If $\xi$ is negative, then $p = a$ is the point closest to $x$.
If $\xi$ is greater than $\norm{d}$, then $p = b$ is the point closest to $x$.
Otherwise $p$ lies between $a$ and $b$:
\[
  p = a + \scalar{\frac{d}{\norm{d}}}{r}\frac{d}{\norm{d}} 
    = a + \frac{\scalar{r}{d}}{\norm{d}^2}d
\]
Notice that no expensive square-root operations are required so far. 
If we now wanted to compute distance from $x$ to $[a, b]$, we would just have
to compute $\norm{x - p}$, 
this can be done numerically stable with the \verb|hypot| function.

In the code, this looks as follows:
\begin{lstlisting}
/**
 * For a point `(x, y)` and a line segment `a` to `b`, computes a
 * point `p` on the line-segment `[a, b]` that is closest to `(x, y)`.
 */
void projectToLineSegment(
    double ax, double ay,
    double bx, double by,
    double x, double y,
    double &px, double &py
) {
  // `d` is the direction of the line-segment
  double dx = bx - ax;
  double dy = by - ay;

  // `r` is the position of `(x,y)` relative to `a`
  double rx = x - ax;
  double ry = y - ay;

  double rdDot = rx * dx + ry * dy;
  double dNormSq = dx * dx + dy * dy;

  // `p` is the closest point to `(x,y)` on the line segment 
  if (rdDot <= 0 || dNormSq == 0) {
    px = ax;
    py = ay;
  } else if (rdDot >= dNormSq) {
    px = bx;
    py = by;
  } else {
    double f = rdDot / dNormSq;
    px = ax + f * dx;
    py = ay + f * dy;
  }
}
\end{lstlisting}
Notice that the code handles the corner-cases where $d = 0$, which is 
important to make the code work with polygons regardless whether the first
vertex is equal to the last vertex or not.

\exercise{3.2}{Texture mapping}
The implementation of parts 3.2 (a-e) can be seen in the code, 
the main functionality is focused in the constructor of \verb|CGView|
where the texture coordinates are generated, and in \verb|CGView::paintGL|,
where the grid is rendered. Notice that we slightly modified the task:
we decided to render an almost-transparent version of the image in the 
background when the grid-mode is activated. Furthermore we avoided the
usage of deprecated \verb|GL_QUADS|, and instead achieved the same 
functionality by using simple lines and triangles.

For the part (f) we need to find a suitable diffeomorphism $\Psi_{r, \alpha}$ 
of a disk $D_r$ of the radius $r$ into itself in order to implement 
local inflation/deflation of the image. 
We decided to compose 
$\Psi_{r, \alpha}$ out of three functions. 
First, consider the following diffeomorphism between the
disk $D_r$ and the whole plane $\Real^2$ (or $\Real^n$, the dimension doesn't 
matter here):
\begin{align*}
  \phi_r: \Real^2 \to D_r       \qquad & \phi_r(x) := \frac{rx}{1 + \norm{x}} \\
  \phi_r^{-1}:  D_r \to \Real^2 \qquad & \phi_r^{-1}(x)  := \frac{x}{r - \norm{x}}.
\end{align*}
This diffeomorphism allows us to go from disk to the plane and back.
Since the plane carries a vector space structure, the \emph{scaling} operation
makes sense there. So we just go from disk to the plane, scale all points by
some fixed factor $\alpha$ and then go back to the disk:
\begin{align*}
  \Psi_{r, \alpha} & := \phi_r\circ(\alpha \cdot -) \circ \phi_r^{-1}.
\end{align*}
For the last part of the whole exercise, we will need the inverse of $\Psi_{r, \alpha}$,
this is obviously given by an analogous operation, 
but this time we scale with $\alpha^{-1}$, that is:
\begin{align*}
  \Psi_{r, \alpha}^{-1} = \Psi_{r, \alpha^{-1}}.
\end{align*}
We have included a possibility to change the brush size with ``+'' and ``-'', as
well as to reset the resulting mess with \verb|Ctrl+R|.
The inflation/deflation effect is demonstrated on the portrait of Escher.

\begin{minipage}[t]{0.45\linewidth}
  \centering\includegraphics[width=0.98\linewidth]{images/escherOriginal.png}
  \captionof{figure}{
    Original portrait of Maurits Cornelis Escher with a grid showing the
    subdivision into small polygons.
  }
  \label{originalEscher}
\end{minipage}
\hspace{0.1\linewidth}
\begin{minipage}[t]{0.45\linewidth}
  \centering\includegraphics[width=0.98\linewidth]{images/escherWarped.png}
  \captionof{figure}{
    Locally warped portrait of Escher, producing a ``fish-eye'' effect.
  }
  \label{warpedEscher}
\end{minipage}

For the deformation in (h-i), we decided to use the distance in texture coordinates,
not in world-coordinates. 
This has the effect that the whole grid behaves more like a sheet of elastic
material, and less like a lump of some very viscous sticky liquid. 
In particular, clicking in $y$ and
dragging to $x$ produces the inverse transformation of clicking in $x$ and then 
dragging to $y$. In contrast, if one uses the distance in world-coordinates, then two
points that once overlap (i.e. have the same world coordinates) can not be separated again.

The distance is computed as
\[
  r := \frac{2}{N} \sqrt{dr^2 + dc^2}  
\]
where $dr$ and $dc$ are integer \emph{index} differences of a grid vertex and the dragged vertex, 
$2$ is the width of the originally rendered rectangle, and $N$ is the number of vertex points.
Then we weight the offset of the picked point depending on the distance:
\[
  r \mapsto \exp\rPar{{-\frac{r^2}{\sigma^2}}}
\]
where we used the brush width as $\sigma$ to make the usage more or less intuitive.
Escher seems happy with this choice of the weight function:

\centering\includegraphics[width=0.49\linewidth]{images/escherSmile.png}

\end{document}
