\documentclass{scrartcl}

\input{./packages}
\input{./definitions}
\newcommand{\exercise}[2]{\vspace{1em}\noindent{\bf Exercise #1 (#2)}}
\renewcommand{\proof}{\vspace{0.8em}\noindent{\bf Proof: }}

\begin{document}
\noindent{\footnotesize Computer Graphics 2014/15, Exercise 3} 
\hfill 
{\footnotesize \input{./currentDate.txt}}
\newline
{\footnotesize \input{../../NAMES.txt}}

\noindent\hrulefill

\exercise{3.1}{Polygon picking and snapping}
(Possibly other group members want to submit something on a sheet of paper?)

\exercise{3.2}{Texture mapping}
The implementation of parts 3.2 (a-e) can be seen in the code, 
the main functionality is focused in the constructor of \verb|CGView|
where the texture coordinates are generated, and in \verb|CGView::paintGL|,
where the grid is rendered. Notice that we slightly modified the task:
we decided to render an almost-transparent version of the image in the 
background when the grid-mode is activated. Furthermore we avoided the
usage of deprecated \verb|GL_QUADS|, and instead achieved the same 
functionality by using simple lines and triangles.

For the part (f) we need to find a suitable diffeomorphism $\Psi_{r, \alpha}$ 
of a disk $D_r$ of the radius $r$ into itself in order to implement 
local inflation/deflation of the image. 
We decided to compose 
$\Psi_{r, \alpha}$ out of three functions. 
First, consider the following diffeomorphism between the
disk $D_r$ and the whole plane $\Real^2$ (or $\Real^n$, the dimension doesn't 
matter here):
\begin{align*}
  \phi_r: \Real^2 \to D_r       \qquad & \phi_r(x) := \frac{rx}{1 + \norm{x}} \\
  \phi_r^{-1}:  D_r \to \Real^2 \qquad & \phi_r^{-1}(x)  := \frac{x}{r - \norm{x}}.
\end{align*}
This diffeomorphism allows us to go from disk to the plane and back.
Since the plane carries a vector space structure, the \emph{scaling} operation
makes sense there. So we just go from disk to the plane, scale all points by
some fixed factor $\alpha$ and then go back to the disk:
\begin{align*}
  \Psi_{r, \alpha} & := \phi_r\circ(\alpha \cdot -) \circ \phi_r^{-1}.
\end{align*}
For the last part of the whole exercise, we will need the inverse of $\Psi_{r, \alpha}$,
this is obviously given by an analogous operation, 
but this time we scale with $\alpha^{-1}$, that is:
\begin{align*}
  \Psi_{r, \alpha}^{-1} = \Psi_{r, \alpha^{-1}}.
\end{align*}
We have included a possibility to change the brush size with ``+'' and ``-'', as
well as to reset the resulting mess with \verb|Ctrl+R|.
The inflation/deflation effect is demonstrated on the portrait of Escher.

\begin{minipage}[t]{0.45\linewidth}
  \centering\includegraphics[width=0.98\linewidth]{images/escherOriginal.png}
  \captionof{figure}{
    Original portrait of Maurits Cornelis Escher with a grid showing the
    subdivision into small polygons.
  }
  \label{originalEscher}
\end{minipage}
\hspace{0.1\linewidth}
\begin{minipage}[t]{0.45\linewidth}
  \centering\includegraphics[width=0.98\linewidth]{images/escherWarped.png}
  \captionof{figure}{
    Locally warped portrait of Escher, producing a ``fish-eye'' effect.
  }
  \label{warpedEscher}
\end{minipage}

For the deformation in (h-i), we decided to use the distance in texture coordinates,
not in world-coordinates. 
This has the effect that the whole grid behaves more like a sheet of elastic
material, and less like a lump of some very viscous sticky liquid. 
In particular, clicking in $y$ and
dragging to $x$ produces the inverse transformation of clicking in $x$ and then 
dragging to $y$. In contrast, if one uses the distance in world-coordinates, then two
points that once overlap (i.e. have the same world coordinates) can not be separated again.

The distance is computed as
\[
  r := \frac{2}{N} \sqrt{dr^2 + dc^2}  
\]
where $dr$ and $dc$ are integer \emph{index} differences of a grid vertex and the dragged vertex, 
$2$ is the width of the originally rendered rectangle, and $N$ is the number of vertex points.
Then we weight the offset of the picked point depending on the distance:
\[
  r \mapsto \exp\rPar{{-\frac{r^2}{\sigma^2}}}
\]
where we used the brush width as $\sigma$ to make the usage more or less intuitive.
Escher seems happy with this choice of the weight function:

\centering\includegraphics[width=0.49\linewidth]{images/escherSmile.png}

\end{document}
