\documentclass{scrartcl}

\usepackage{amsmath}	  % required for math in general
\usepackage{amsthm}     % environments for theorems, qed's etc
                        % (loaded after amsmath)
\usepackage{amssymb}	  % doublestroke symbols, other mathematical symbols
%\usepackage{dsfont}     % required for double-stroke 1 as characteristic function
\usepackage{array}	    % control of matrices and tables
\usepackage{graphicx}   % images

\usepackage{enumitem}   % more fine-grained control over enumerations
\setdescription{leftmargin=\parindent,labelindent=\parindent}

\usepackage{listings} % code listings
\lstset{basicstyle=\ttfamily\scriptsize}

% \input{diagrams.sty} (no category theory this time)

\usepackage{helvet}   % use (much fresher looking) helvetica for everything
\renewcommand{\familydefault}{\sfdefault}

\usepackage[weather]{ifsym}      % \Lightning symbol
% \usepackage{mathabx}             % \Asterisk causes some conflicts

% forcing the fucking floats to stop fucking floating like a fucking piece of
% shit in an ocean of fucking shit
\renewcommand{\topfraction}{.85}
\renewcommand{\bottomfraction}{.7}
\renewcommand{\textfraction}{.15}
\renewcommand{\floatpagefraction}{.66}
\renewcommand{\dbltopfraction}{.66}

% making all references into hyperlinks
\usepackage[dvipsnames]{xcolor}
\usepackage{hyperref}

\hypersetup{colorlinks=true,linkcolor=MidnightBlue,pdfborderstyle={/W 0}}

%\usepackage{anyfontsize}
\usepackage{datetime}

% forall
\let\oldforall\forall
\renewcommand{\forall}{\oldforall\,}

% parentheses
\newcommand{\rPar}[1]{\left(#1\right)} % round parens
\newcommand{\sPar}[1]{\left[#1\right]} % square parens
\newcommand{\cPar}[1]{\left\{#1\right\}} % curved parens 
\newcommand{\aPar}[1]{\left\langle #1 \right\rangle} % angle brackets

% floor and ceiling
\newcommand{\floor}[1]{{\left\lfloor#1\right\rfloor}} % curved parens 
\newcommand{\ceil}[1]{{\left\lceil#1\right\rceil}} % curved parens 

% norms
\newcommand{\abs}[1]{\left\lvert #1\right\rvert}
\newcommand{\norm}[1]{\left\lVert #1\right\rVert}
\newcommand{\scalar}[2]{\left\langle#1,#2\right\rangle}
\newcommand{\cross}{\times}
\DeclareMathOperator{\diam}{diam}
\DeclareMathOperator{\B}{B}

% intervals
\newcommand{\openOpenInterval}[2]{{\left(#1,#2\right)}}
\newcommand{\openClosedInterval}[2]{{\left(#1,#2\right]}}
\newcommand{\closedOpenInterval}[2]{{\left[#1,#2\right)}}
\newcommand{\closedClosedInterval}[2]{{\left[#1,#2\right]}}

% restriction of functions
\newcommand{\restrict}[2]{{\left.#1\right\vert_{#2}}}

% numbers
\newcommand{\Natural}{\mathbb{N}}
\newcommand{\Integer}{\mathbb{Z}}
\newcommand{\Real}{\mathbb{R}}
\newcommand{\Rational}{\mathbb{Q}}
\newcommand{\PositiveReal}{\Real_{>0}}
\newcommand{\NonnegativeReal}{\Real_{\geq0}}
\newcommand{\Complex}{\mathbb{C}}
\renewcommand{\i}{i}
\newcommand{\Quaternion}{\mathbb{H}}
\newcommand{\Boolean}{\mathbb{B}}

% function spaces
\newcommand{\SemiLebesgue}{\mathscr{L}}
\newcommand{\Continuous}{C}
\newcommand{\Lebesgue}{L}
\newcommand{\Sobolev}{H}
\newcommand{\Hilbert}{\mathscr{H}}
\newcommand{\Schwarz}{\mathscr{S}}

% set
\newcommand{\setPredicate}[2]{{\left\{#1\,\left\vert\, #2\right.\right\}}}
\newcommand{\set}[1]{{\left\{#1\right\}}}
\newcommand{\cardinality}[1]{\left\lvert #1 \right\rvert}
\newcommand{\powerset}[1]{\mathfrak{P}\left(#1\right)}
\DeclareMathOperator*{\intersection}{\bigcap}
\DeclareMathOperator*{\union}{\bigcup}
\newcommand{\disjointUnion}{\biguplus}
\renewcommand{\complement}[1]{#1^c}
% \newcommand{\setminus}{\backslash}
\newcommand{\injective}{\hookrightarrow}
\newcommand{\surjective}{\twoheadrightarrow}
%\DeclareMathOperator{\ker}{ker} % already exists... im does not?
\DeclareMathOperator{\im}{im}

% topological operators
\DeclareMathOperator{\Cl}{Cl}
\newcommand{\Closure}[2]{\Cl_{#1}\left(#2\right)}
\DeclareMathOperator{\const}{const}

% span and conv
\DeclareMathOperator*{\conv}{conv}
\DeclareMathOperator*{\linhull}{span}

% matrices
\newcommand{\mat}[2]{\left[\begin{array}{#1}#2\end{array}\right]}
\DeclareMathOperator*{\diag}{diag}

% landau symbols
\newcommand{\LandauO}[1]{\mathcal{O}\left(#1\right)}

% derivatives
\newcommand{\dd}[2]{\frac{\partial #1}{\partial #2}}
\newcommand{\differential}[1]{\boldsymbol{D}_{#1}}

% integrals
\renewcommand{\d}{\quad\mathrm{d}}

% characteristic functions, expected values, variances, covariances
% stochastic stuff
\newcommand{\one}[1]{\mathds{1}_{#1}}
\newcommand{\weakconv}[1]{\overset{#1}{\Longrightarrow}}
\newcommand{\wlim}{\mathop{\mathrm{wlim}}}
\newcommand{\vlim}{\mathop{\mathrm{vlim}}}

% lim inf lim sup
% \DeclareMathOperator{\liminf}{lim inf}
% \DeclareMathOperator{\limsup}{lim sup}

% qed etc.
\renewcommand{\qedsymbol}{$\blacksquare$}
\newcommand{\result}{\hfill $\Diamond$}

% lattices
\newcommand{\meet}{\wedge}
\newcommand{\join}{\vee}
\newcommand{\negate}{\neg}

% listings: Scala
\lstdefinelanguage{scala}{
  morekeywords={abstract,case,catch,class,def,%
    do,else,extends,false,final,finally,%
    for,if,implicit,import,match,mixin,%
    new,null,object,override,package,%
    private,protected,requires,return,sealed,%
    super,this,throw,trait,true,try,%
    type,val,var,while,with,yield},
  otherkeywords={=>,<-,<\%,<:,>:,\#,@},
  sensitive=true,
  morecomment=[l]{//},
  morecomment=[n]{/*}{*/},
  morestring=[b]",
  morestring=[b]',
  morestring=[b]"""
}
\lstset{showstringspaces=false}

% making references look a little nices
\let\oldRef\ref
\renewcommand{\ref}[1]{(\oldRef{#1})}

% weird stuff for computer science
\DeclareMathOperator{\arity}{ar}

% cat, category theory
% Bunch of categories
\DeclareMathOperator{\Id}{Id}
\DeclareMathOperator{\Top}{Top}
\DeclareMathOperator{\hTop}{h-Top}
\DeclareMathOperator{\Sets}{Sets}
\DeclareMathOperator{\Rel}{Rel}
\DeclareMathOperator{\FinSets}{FinSets}
\DeclareMathOperator{\Grp}{Grp}
\DeclareMathOperator{\Cat}{Cat}
\DeclareMathOperator{\Grpd}{Grpd}
\newcommand{\cat}[1]{\mathcal{#1}}
\newcommand{\Obj}{\mathrm{Obj}}
\newcommand{\Hom}{\mathrm{Hom}}
\newcommand{\op}{\mathrm{op}}
\newcommand{\nat}{\xrightarrow{\bullet}}
\newcommand{\iso}{\cong}
\newcommand{\dom}{\mathrm{dom}}
\newcommand{\cod}{\mathrm{cod}}
\DeclareMathOperator{\coeq}{Coeq}
\newcommand{\fst}{\mathrm{fst}}
\newcommand{\snd}{\mathrm{snd}}
\DeclareMathOperator{\Aut}{Aut}
\DeclareMathOperator{\End}{End}

% functors frequently used in various contexts
\DeclareMathOperator{\Free}{Free}
\DeclareMathOperator{\Forget}{Forget}

% empty set that is round
\let\emptyset\varnothing

% generated groups
\newcommand{\gen}[1]{\left\langle#1\right\rangle}
\newcommand{\normalSub}{\triangleleft}
\newcommand{\Asterisk}{\mathop{\scalebox{1.5}{\raisebox{-0.2ex}{$\ast$}}}}
\newcommand{\Sym}{\mathrm{Sym}}

% argmax argmin argsup etc.
\DeclareMathOperator{\argsup}{argsup}

% number theoretic operators
\DeclareMathOperator{\lcm}{lcm}

% get rid of the ugly-looking "epsilon"
\renewcommand{\epsilon}{\varepsilon}

% get rid of the empty-looking "angle"
\renewcommand{\angle}{\measuredangle}

\newcommand{\exercise}[2]{\vspace{1em}\noindent{\bf Exercise #1 (#2)}}
\renewcommand{\proof}{\vspace{0.8em}\noindent{\bf Proof: }}

\begin{document}
\noindent{\footnotesize Computer Graphics 2014/15, Exercise 7} 
\hfill 
{\footnotesize 14.01.2015}
\newline
{\footnotesize \input{../../NAMES.txt}}

\noindent\hrulefill

\exercise{7.1}{SLERP}
Let $(X, \scalar{-}{-})$ be an arbitrary scalar product space
and $\norm{-}$ the induced norm:
\[
  \norm{x} := \sqrt{\scalar{x}{x}}.
\]
In particular, $X$ can be $\Real^4 \iso \Quaternion$ with the
standard scalar product.

Let $a,b\in X$ be two different unit vectors and $\phi$ the 
positive angle from $(0, 2\pi)$ such that 
$\cos(\phi) = \scalar{a}{b}$.
Consider the following function:
\[
    slerp(a, b, t) := 
    \frac{\sin\rPar{(1-t)\phi}}{\sin{\phi}} a +
    \frac{\sin\rPar{t\phi}}{\sin{\phi}} b,
\]
where $t\in [0, 1]$.
We claim that this is a linear angular interpolation
between $a$ and $b$.

\noindent \textbf{Proof: } We define a unit vector orthogonal
to $a$ as follows:
\[
  u := \frac{b - \scalar{a}{b}a}{\norm{b - \scalar{a}{b}a}},
\]
this can be seen as a single step of Gram-Schmidt
orthogonalization on the basis $\set{a, b}$.
We want to simplify the denominator, so we compute:
\[
  \norm{b - \scalar{a}{b}a}^2 
  = 1 - 2\scalar{a}{b}^2 + \scalar{a}{b}^2
  = 1 - \rPar{\cos(\phi)}^2 
  = \rPar{\sin(\phi)}^2,
\]
this allows us to express $u$ as follows:
\[
  u = \frac{b - \cos(\phi)a}{\sin(\phi)}.
\]
Now $\set{a, u}$ is an orthonormal basis, so we can 
express $b$ as follows:
\[
  b = \scalar{a}{b}a + \scalar{u}{b}u.
\]
The first coordinate is by definition just $\cos(\phi)$,
the second coordinate can be computed as follows:
\[
  \scalar{u}{b} 
  = \frac{1}{\sin(\phi)}\scalar{b - \cos(\phi)a}{b}
  = \frac{1}{\sin(\phi)}\rPar{\norm{b}^2 - (\cos(\phi))^2} 
  = \sin(\phi).
\]
This gives us a simple description of $b$ in terms of
$a$ and $u$:
\[
  b = \cos(\phi)a + \sin(\phi)u.
\]
Now all we have to do is to plug it into the $slerp$ formula:
\begin{align*}
  slerp(a, b, t) 
    &= \frac{\sin((1-t)\phi)}{\sin(\phi)} a + 
       \frac{\sin(t\phi)}{\sin(\phi)} b \\
    &= \frac{\sin((1-t)\phi) + \sin(t\phi)\cos(\phi)}{\sin(\phi)}
       a + \sin(\phi t) u \\
    &= \frac{-\cos(\phi)\sin(t\phi) + \sin(\phi)\cos(t\phi)
              + \sin(t\phi)\cos(\phi)
        }{\sin(\phi)} a  + \sin(t\phi) u \\
    &= \cos(t\phi)a + \sin(t\phi)u.
\end{align*}
Here we used the addition theorem for $\sin$ in the third line.
\hfill \qed

\exercise{7.2}{SLERP + trackball} See code.
Note that $slerp$ and Euler angle interpolation is activated by 
pressing S or E.

\exercise{7.3}{Symmetry: technical details of a proof}
Let $u_i,v_i\in \Real^3$ for all $i\in 0 \dots n$ for some
$n\in\Natural$.
We consider the following matrices:
\[
  U_i = \mat{cc}{0 & -u_i^T \\ u_i & -[u_i]_\cross}
  \quad
  V_i = \mat{cc}{0 & -v_i^T \\ v_i & [v_i]_\cross}
\]
and claim that 
\[
  \sum_{i=1}^n U_i^T V_i
\]
is symmetric.
For this it is obviously sufficient to show that each 
summand $U_i^T V_i$ is symmetric. Let $u\in\set{u_i}_i$
and $v\in\set{v_i}_i$ be some vectors and $U, V$ matrices
as above, but without the indices.

It holds:
\begin{align*}
  \mat{cc}{0 & -u^T \\ u & -[u]_\cross}^T
  \mat{cc}{0 & -v^T \\ v & [v]_\cross} 
  &= 
  \mat{cc}{0 & u^T \\ -u & [u]_\cross}
  \mat{cc}{0 & -v^T \\ v & [v]_\cross} \\
  &= 
  \mat{cc}{
    u^T v & (u\cross v)^T \\
    u\cross v & uv^T + vu^T + u^Tv I
  }
\end{align*}
Here we used the following identity in first row, second column:
\[
  u^T[v]_\cross = ([v]_\cross^Tu)^T = 
  (-v\cross u)^T = (u\cross v)^T 
\]
and the matrix-version of the BAC-CAB-rule in the lower right 
corner:
\[
  [a]_\cross[b]_\cross = ba^T - b^Ta I.
\]
The resulting matrix is obviously symmetric, therefore 
the sum of such symmetric matrices is also symmetric.
\hfill \qed

\exercise{7.4}{Data registration / Image registration}
Code incomplete. No matrix addition or eigenvector implementation
found. Please tell to use something like MATLAB next time the 
homework is not solvable with available tools.

\end{document}
