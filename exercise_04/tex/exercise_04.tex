\documentclass{scrartcl}

\input{./packages}
\input{./definitions}
\newcommand{\exercise}[2]{\vspace{1em}\noindent{\bf Exercise #1 (#2)}}
\renewcommand{\proof}{\vspace{0.8em}\noindent{\bf Proof: }}

\begin{document}
\noindent{\footnotesize Computer Graphics 2014/15, Exercise 4} 
\hfill 
{\footnotesize \input{./currentDate.txt}}
\newline
{\footnotesize \input{../../NAMES.txt}}

\noindent\hrulefill

\exercise{4.1}{ Parallel-perspective point-picking } 
(See code)

\exercise{4.2}{ Cavalier perspective } 
 
\begin{figure}[!h]
	\centering\includegraphics[width=0.4\linewidth,height=0.15\linewidth]{images/Trans.pdf}
\end{figure}To get the point $p'=\begin{pmatrix}
x' \\ y' \\ 0\\1
\end{pmatrix}$ from the point $p=\begin{pmatrix}
x \\ y \\ z\\1
\end{pmatrix}$ we must divide z in half and project it on the x/y-axis and must add this to the x/y value.
Projection:
\begin{figure}[!h]
	\centering\includegraphics[width=0.3\linewidth,height=0.3\linewidth]{images/projection.pdf}
\end{figure}
\\The z' in the picture equals -0.5z $\Rightarrow p'=\begin{pmatrix}
x-0.5cos(\alpha)z\\
y-0.5cos(\alpha)z\\
0\\
1
\end{pmatrix}
$
\\$p'=M p\Longleftrightarrow \begin{pmatrix}
x-0.5cos(\alpha)z\\
y-0.5cos(\alpha)z\\
0\\
1
\end{pmatrix}= \begin{pmatrix}
a & b& c& d\\
e& f& g& h\\
i& j & k& l\\
m & n& o& q
\end{pmatrix}
\begin{pmatrix}
x\\
y\\
z\\
1
\end{pmatrix}=\begin{pmatrix}
ax+by+cz+d\\
ex+fy+gz+h\\
ix+jy+kz+l\\
mx+ny+oz+q
\end{pmatrix}
$
\\$\Rightarrow M= \begin{pmatrix}
1 & 0& -0.5 \cos(\alpha)& 0\\
0& 1& -0.5 \cos(\alpha)& 0\\
0& 0 & 0& 0\\
0 & 0& 0& 1
\end{pmatrix}$
\\ With $\alpha=45^\circ$ and $\cos(45^\circ)=\sin(45^\circ)=\frac{1}{\sqrt{2}}$ we get:
 $M= \begin{pmatrix}
1 & 0& -\frac{1}{2\sqrt{2}}& 0\\
0& 1& -\frac{1}{2\sqrt{2}}& 0\\
0& 0 & 0& 0\\
0 & 0& 0& 1
\end{pmatrix}$

\exercise{4.3}{ Perspective projection and lines } 
We want to understand the effects of perspective projection on lines in three dimensional
space.
Let $f>0$ be some constant. 
We omit the discussion of all the tricks with the third component of the projection (
which are relevant only for the z-buffer),
and we also omit all the details related to hardware-specific screen sizes and sign-conventions.
Instead, we assume without loss of generality, that the observer is located at $0$ and
is looking into the positive $z$-direction, projecting everything on a near-clipping
plane at $z=f$. That means, that a point $(x,y,z)$ is projected to $(fx/z, fy/z, f)$.
Expressed as a matrix for homogeneous coordinates, this looks as follows:
\[
  \mat{cccc}{
     f & 0 & 0 & 0 \\
     0 & f & 0 & 0 \\
     0 & 0 & f & 0 \\
     0 & 0 & 1 & 0 
  }
  \mat{c}{
    x \\
    y \\
    z \\
    1
  }
  = 
  \mat{c}{
    fx \\
    fy \\
    fz \\
    z 
  } 
  \rightharpoonup 
  \mat{c}{
    fx/z \\ 
    fy / z \\
    f
  }
\]
Now let $b\in \Real^3$ be some base vector and $d\in \Real^3$ be some dimension vector, we 
want to consider the straight line $\setPredicate{b + t d}{t\in\Real}$.
Applying the projection to the points of the straight line, we obtain:
\[
  \mat{cccc}{
     f & 0 & 0 & 0 \\
     0 & f & 0 & 0 \\
     0 & 0 & f & 0 \\
     0 & 0 & 1 & 0 
  }
  \mat{c} {
    b_x + t d_x \\
    b_y + t d_y \\
    b_z + t d_z \\
    1
  }
  \rightharpoonup
  \mat{c}{
    f \frac{b_x + t d_x}{b_z + t d_z} \\
    f \frac{b_y + t d_y}{b_z + t d_z} \\
    f
  }
\]
Now we want to compute the derivative of the right-hand side with respect to $t$.
Obviously, it is sufficient to consider only the first component (the 
second looks almost the same, the third is a constant). Let's also drop the
factor $f$ for a moment. It holds:
\begin{align*}
  \frac{d}{dt} \frac{b_x + t d_x}{b_z + t d_z} 
    &= \frac{d_x}{b_z + t d_z} - \frac{(b_x + t d_x) d_z}{(b_z + t d_z)^2} \\
    &= \frac{d_x(b_z + t d_z) - (b_x + t d_x) d_z}{(b_z + t d_z)^2} \\
    &= \frac{d_x b_z - b_x d_z}{(b_z + t d_z)^2}
\end{align*}
For all three components together, it means:
\begin{align*}
  \frac{d}{dt}\mat{c}{
    f \frac{b_x + t d_x}{b_z + t d_z} \\
    f \frac{b_y + t d_y}{b_z + t d_z} \\
    f
  } = 
  \frac{f}{(b_z + t d_z)^2}
  \mat{c}{
     d_x b_z - b_x d_z \\
     d_y b_z - b_y d_z \\
     0
  }.
\end{align*}
Now we can make the following observations.
First, notice that although the whole velocity vector is not constant, 
\emph{it's direction is}: it is always the vector $(
     d_x b_z - b_x d_z,
     d_y b_z - b_y d_z, 0
  )$, scaled by some $t$-dependent factor. 
  In particular, the projection of the line is contained in the 
  line 
  $\setPredicate{(f b_x / b_z, f b_y / b_z, f) + q (d_x b_z - b_x d_z,
  d_y b_z - b_y d_z, 0)}{q \in \Real}$. 
  If one does not care too much 
  about the fact that the original line is infinite, while the projected 
  line is bounded by vanishing points, one could express it as ``
  straight lines are projected to straight lines'' (this is the statement (i)).
  
  Now take a closer look at the direction vector $(d_x b_z - b_x d_z,
  d_y b_z - b_y d_z, 0)$. As long as $d_z$ is not zero, this direction 
  vector depends on the coordinates of the base point $b_x$ and $b_y$,
  so that parallel lines with $d_z\neq 0$ will in general no
  longer be parallel after projection (this shows (ii)).

  However, if $d_z$ is zero, then
  the direction vector becomes just $b_z(d_x, d_y, 0)$ after projection, that is, it is a scaled version of the original vector $(d_x, d_y, d_z)$. This means that
  parallel lines that are parallel to the projection plane remain parallel, 
  and do not intersect in a vanishing point (iii).
\end{document}
